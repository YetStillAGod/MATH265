\documentclass{article}
\usepackage[utf8]{inputenc}
\usepackage{setspace}
\usepackage{tikz}
\usetikzlibrary{positioning}
\usepackage{amsfonts}
\usepackage{amssymb}
\usepackage{amsmath}
\usepackage{amsthm}
\usepackage{systeme}
\usepackage{mathtools}
\usepackage{hyperref}
\usepackage{venndiagram}
\usepackage{pgfplots}
\usetikzlibrary{pgfplots.statistics}
\pgfplotsset{compat=newest}

\allowdisplaybreaks

\begin{document}

\section*{Question 1}

~

\subsection*{a}

~

\begin{proof}
    \begin{align*}
        &\forall n\in \mathbb{Z} _+, \frac{1}{n}\geqslant0\\
        \Rightarrow&1-\frac{1}{n}\leqslant1\\
        &2|n+1\implies (-1)^n=-1\\
        \Rightarrow&(-1)^n(1-\frac{1}{n})<1\\
        &2|n\implies  (-1)^n=1\\
        \Rightarrow&(-1)^n(1-\frac{1}{n})\leqslant 1\\
        &\\
        \Rightarrow&(-1)^n(1-\frac{1}{n})\leqslant 1\\
        \Rightarrow&1\text{ is an upper bound of }S\\
    \end{align*}
\end{proof}

~

\subsection*{b}

~

\begin{proof}
    \begin{align*}
        &\text{Suppose}:\exists m<1\text{ is an upper bound of }S\\
        \Rightarrow&\exists n\in \mathbb{R} _+:m= 1-\frac{1}{n}\\
        \Rightarrow&\exists a,b\in\mathbb{Z} _+:a\leqslant n< b\\
        \Rightarrow&1-\frac{1}{a}\leqslant 1-\frac{1}{n}< 1-\frac{1}{b}\\
        &1-\frac{1}{b}\in S\\
        \Rightarrow&\exists x\in S:m< x\\
        \Rightarrow&m\text{ is not an upper bound}\nLeftrightarrow m\text{ is an upper bound}\\
        \Rightarrow&m\text{ is an upper bound of }S\implies m\geqslant 1\\
    \end{align*}
\end{proof}

~

\subsection*{c}

~

\begin{proof}
    \begin{align*}
        &\text{b}:\forall m\text{ is an upper bound of }S\implies m\geqslant 1\\
        &\text{a}:1\text{ is an upper bound of }S\\
        \Rightarrow&1\text{ is the least upper bound of }S\\
        \Rightarrow&\sup S=1\\
    \end{align*}
\end{proof}

\newpage

\section*{Question 2}

~

\begin{proof}
    \begin{align*}
        &\text{case 1}:\sup(A+B)\leqslant\sup(A)+\sup(B)\\
        &A+B=\{a,b|a\in A,b\in B\}\\
        &c\in A+B\coloneqq a+b,a\in A,b\in B\\
        &\forall a\in A, a\leqslant \sup A, \forall b\in B, b\leqslant \sup B\\
        \Rightarrow&\forall a\in A, b\in B, c\leqslant \sup A+\sup B\\
        \Rightarrow&\sup A+\sup B\text{ is an upper bound of }A+B\\
        \Rightarrow&\sup(A+B)\leqslant \sup A+\sup B\\
        &\\
        &\text{case 2}:\sup A+\sup B\leqslant \sup(A+B)\\
        &\text{arbitrary }a+b\in A+B\\
        \Rightarrow&a+b\leqslant \sup(A+B)\\
        &a\leqslant \sup(A+B)-b\\
        \Rightarrow&\forall a\in A,a\leqslant \sup(A+B)-b\\
        \Rightarrow&\sup(A+B)-b\text{ is an upper bound of }A\\
        \Rightarrow&\sup A\leqslant \sup(A+B)-b\\
        &b\leqslant \sup(A+B)-\sup A\\
        \Rightarrow&\forall b\in B,b\leqslant \sup(A+B)-\sup A\\
        \Rightarrow&\sup(A+B)-\sup A\text{ is an upper bound of }B\\
        \Rightarrow&\sup B\leqslant \sup(A+B)-\sup A\\
        &\sup A+\sup B\leqslant \sup(A+B)\\
        &\\
        &\sup(A+B)\leqslant \sup A+\sup B\land \sup A+\sup B\leqslant \sup(A+B)\\
        \Rightarrow&\sup(A+B)=\sup A+\sup B\\
    \end{align*}
\end{proof}

\newpage

\section*{Question 3}

~

\begin{proof}
    \begin{align*}
        &S\coloneqq\{x^n,n\in\mathbb{Z} _+\}\\
        &\text{Supppose}:\exists m:m\text{ is the upper bound of }S\\
        \Rightarrow&\forall n\in\mathbb{Z} _+,m\geqslant x^n\\
        &\log_{x}(m)\geqslant n\\
        &p\coloneqq\lceil\log_{x}(m) \rceil +1\\
        \Rightarrow&p>\log_{x}(m)\\
        \Rightarrow&x^p>x^{\log_{x}(m)}=m\\
        &p=\lceil\log_{x}(m) \rceil +1\\
        \Rightarrow&p\in\mathbb{Z} _+\\
        \Rightarrow&x^p\in S\\
        \Rightarrow&\exists x^p\in S:x^p>m\nleftrightarrow m\text{ is the upper bound of }S\\
        \Rightarrow&S\text{ is not bounded above}\\
    \end{align*}
\end{proof}

\newpage

\section*{Question 4}

~

\subsection*{Approach 1}

~

\begin{proof}
    \begin{align*}
        &A\coloneqq\{a_n,n\in\mathbb{Z} _+\}\\
        &B\coloneqq\{b_n,n\in\mathbb{Z} _+\}\\
        &(1,4)=\bigcup_{n=1}^{\infty}I_n,[2,3]=\bigcap_{n=1}^{\infty}I_n\\
        \Leftrightarrow&\exists I_n=[a_n,b_n):\inf A=1\notin A,\inf B=3\notin B, \sup A=2\in A,\sup B=4\in B\\
        &I_n\coloneqq[1+\frac{1}{n},4-\frac{1}{n})\\
        &A\coloneqq\{1+\frac{1}{n},n\in\mathbb{Z} _+\}\\
        &B\coloneqq\{3+\frac{1}{n},n\in\mathbb{Z} _+\}\\
        &\\
        &\inf A=1\notin A:\\
        &\forall n\in\mathbb{Z} _+,\frac{1}{n}>0\\
        \Rightarrow&1\notin A, \forall a_n\in A,a_n>1\\
        &1\text{ is a lower bound of }A\\
        &\text{Suppose}: \exists p>1: p\text{ is a lower bound of }A\\
        &\exists \varepsilon>0:p=1+\varepsilon\\
        &\varepsilon>0\\
        \Rightarrow&\exists n_1\in\mathbb{Z} _+:\varepsilon<n_1\\
        \Rightarrow&1+\frac{1}{\varepsilon}>1+\frac{1}{n_1}\\
        &1+\frac{1}{n_1}\in A\\
        \Rightarrow&\exists a_i\in A:a_i<p\nLeftrightarrow p\text{ is a lower bound of }A\\
        \Rightarrow&\nexists p>1:p\text{ is a lower bound of }A\\
        \Rightarrow&1\text{ is the most lower bound of }A\\
        &\inf A=1\\
        \Rightarrow&\inf A=1\notin A\\
        &\\
        &\inf B=3\notin B:\\
        &\forall n\in\mathbb{Z} _+,\frac{1}{n}>0\\
        \Rightarrow&3\notin B,\forall b_n\in B, b_n>3\\
        \Rightarrow&3\text{ is a lower bound of }B\\
        &\text{Suppose}:\exists s>3 \text{ is a lower bound of }B\\
        &\exists \varepsilon>0: s=3+\varepsilon\\
        &\varepsilon>0\\
        \Rightarrow&\exists n_4\in\mathbb{Z} _+:\varepsilon<n_4\\
        \Rightarrow&3+\frac{1}{\varepsilon}>3+\frac{1}{n_4}\\
        &3+\frac{1}{n_4}\in B\\
        \Rightarrow&\exists b_i\in B:s>b_i\nLeftrightarrow s \text{ is a lower bound of }B\\
        \Rightarrow&\nexists s>3\text{ is a lower bound of }B\\
        \Rightarrow&3\text{ is the most lower bound of }B\\
        &\inf B=3\\
        \Rightarrow&\inf B=3\notin B\\
        &\\
        & \sup A=2\in A:\\
        &n=1:1+\frac{1}{1}=2\in A\\
        &\forall b\in\mathbb{Z} _+,\frac{1}{n}\leqslant 1\\
        \Rightarrow&1+\frac{1}{n}\leqslant 2\\
        &2\text{ is an upper bound of }A\\
        &\text{Suppose}:\exists m<2\text{ is a upper bound of }A\\
        &\exists \varepsilon<1:m=1+\varepsilon\\
        &\varepsilon<1\\
        &\exists n_3\in\mathbb{Z} _+:\frac{1}{\varepsilon}>n_3\\
        &1+\varepsilon<1+\frac{1}{n_3}\\
        &1+\frac{1}{n_3}\in A\\
        \Rightarrow&\exists a_i\in A:m<a_i\nLeftrightarrow m\text{ is a upper bound of }A\\
        \Rightarrow&\nexists m<2:m\text{ is a upper bound of }A\\
        \Rightarrow&2\text{ is the least upper bound of }A\\
        &\sup A=2\\
        \Rightarrow&\sup A=2\in A\\
        &\\
        &\sup B=4\in B\\
        &n=1:3+\frac{1}{1}=3\in B\\
        &\forall n\in\mathbb{Z} _+,\frac{1}{n}\leqslant 1\\
        \Rightarrow&3+\frac{1}{n}\leqslant 4\\
        &4\text{ is a lower bound of }B\\
        &\text{Suppose}: \exists q<4: q\text{ is an upper bound of }B\\
        &\exists \varepsilon<1:q=3+\varepsilon\\
        &\varepsilon<1\\
        \Rightarrow&\exists n_2\in\mathbb{Z} _+:\frac{1}{\varepsilon}>n_2\\
        \Rightarrow&3+\varepsilon<3+\frac{1}{n_2}\\
        &3+\frac{1}{n_2}\in B\\
        \Rightarrow&\exists b_i\in B:q<b_i\nLeftrightarrow q\text{ is an upper bound of }B\\
        \Rightarrow&\nexists q>3:q\text{ is an upper bound of }B\\
        \Rightarrow&4\text{ is the least upper bound of }B\\
        &\sup B=4\\
        \Rightarrow&\sup B =4\in B\\
        &\\
        \Rightarrow&I_n=[a_n,b_n)=[1+\frac{1}{n},4-\frac{1}{n}):\\
        &(1,4)=\bigcup_{n=1}^{\infty}I_n,[2,3]=\bigcap_{n=1}^{\infty}I_n\\
    \end{align*}
\end{proof}

~

\subsection*{Approach 2}

~

\begin{proof}
    \begin{align*}
        &A\coloneqq\{a_n,n\in\mathbb{Z} _+\}\\
        &B\coloneqq\{b_n,n\in\mathbb{Z} _+\}\\
        &(1,4)=\bigcup_{n=1}^{\infty}I_n,[2,3]=\bigcap_{n=1}^{\infty}I_n\\
        &I_n\coloneqq[1+\frac{1}{n},4-\frac{1}{n})\\
        &A\coloneqq\{1+\frac{1}{n},n\in\mathbb{Z} _+\}\\
        &B\coloneqq\{3+\frac{1}{n},n\in\mathbb{Z} _+\}\\
        &\{1+\frac{1}{n}\}^\infty_{n=1}\rightarrow1\\
        &n=1:3+\frac{1}{n}=4\\
        &\text{Since }\{3+\frac{1}{n}\}^\infty_{n=1}\rightarrow3\\
        \Rightarrow&\max (3+\frac{1}{n})=4\\
        \Rightarrow&(1,4)=\bigcup_{n=1}^{\infty}I_n\\
        &n=1:1+\frac{1}{n}=2\\
        &\text{Since }\{1+\frac{1}{n}\}^\infty_{n=1}\rightarrow1\\
        \Rightarrow&\max (1+\frac{1}{n})=2\\
        &\{3+\frac{1}{n}\}^\infty_{n=1}\rightarrow3\\
        \Rightarrow&[2,3]=\bigcap_{n=1}^{\infty}I_n\\
        \Rightarrow&(1,4)=\bigcup_{n=1}^{\infty}I_n,[2,3]=\bigcap_{n=1}^{\infty}I_n\\
    \end{align*}
\end{proof}
\end{document}