\documentclass{article}
\usepackage[utf8]{inputenc}
\usepackage{setspace}
\usepackage{tikz}
\usetikzlibrary{positioning}
\usepackage{amsfonts}
\usepackage{amssymb}
\usepackage{amsmath}
\usepackage{amsthm}
\usepackage{systeme}
\usepackage{mathtools}
\usepackage{hyperref}
\usepackage{venndiagram}

\begin{document}
\section*{Question 1}

\begin{proof}
    \begin{align*}
        &n=1:\\
        &\frac{1}{\sqrt[3]{1}}=1\geqslant1^{\frac{2}{3}}\\
        &\text{Suppose: }\sum_{n=1}^{k}\frac{1}{\sqrt[3]{n}}\geqslant k^{\frac{2}{3}}\\
        &k+1:\\
        &LHS=\sum_{n=1}^{k+1}\frac{1}{\sqrt[3]{n}}\\
        &RHS=(k+1)^{\frac{2}{3}}\\
        &\Delta LHS=\frac{1}{\sqrt[3]{k+1}}=(k+1)^{-\frac{1}{3}}\\
        &\Delta RHS=(k+1)^{\frac{2}{3}}-k^{\frac{2}{3}}\\
        &\frac{\Delta RHS}{\Delta LHS}=((k+1)^{\frac{2}{3}}-k^{\frac{2}{3}})(k+1)^{\frac{1}{3}}\\
        &=k+1-(k+1)^{\frac{1}{3}}k^{\frac{2}{3}}\\
        &k<(k+1)^{\frac{1}{3}}k^{\frac{2}{3}}<k+1\\
        \Rightarrow&0<k+1-(k+1)^{\frac{1}{3}}k^{\frac{2}{3}}<1\\
        &0<\frac{\Delta RHS}{\Delta LHS}<1\\
        \Rightarrow&\Delta LHS>\Delta RHS\\
        &\sum_{n=1}^{k}\frac{1}{\sqrt[3]{n}}\geqslant k^{\frac{2}{3}}\\
        \Rightarrow&\sum_{n=1}^{k}\frac{1}{\sqrt[3]{n}}+\Delta LHS\geqslant k^{\frac{2}{3}}+\Delta RHS\\
        \Rightarrow&\sum_{n=1}^{k+1}\frac{1}{\sqrt[3]{n}}\geqslant(k+1)^{\frac{2}{3}}\\
        &\frac{1}{\sqrt[3]{1}}\geqslant1^{\frac{2}{3}}\land (\sum_{n=1}^{k}\frac{1}{\sqrt[3]{n}}\geqslant k^{\frac{2}{3}}\implies \sum_{n=1}^{k+1}\frac{1}{\sqrt[3]{n}}\geqslant(k+1)^{\frac{2}{3}})\\
        \Rightarrow&\forall n\in \mathbb{N} :\sum_{i=1}^{n}\frac{1}{\sqrt[3]{i}}\geqslant n^{\frac{2}{3}}\\
    \end{align*}
\end{proof}

\newpage

\section*{Question 2}

~

\begin{proof}
    \begin{align*}
        &n=0:\\
        &x_1=\frac{1}{8}{x_0}^2+2\\
        &=\frac{9}{8}+2=3\frac{1}{8}\\
        &3<3\frac{1}{8}<4\\
        &3<x_1<4\\
        &\text{Suppose: }x_k<x_{k+1}<4\\
        &k+1:\\
        &x_{k+2}=\frac{1}{8}{x_{k+1}}^2+2\\
        &x_{k+2}-x_{k+1}=\frac{1}{8}{x_{k+1}}^2-x_{k+1}+2=\frac{1}{8}(x_{k+1}-4)^2\\
        &x_{k+1}<4\\
        \Rightarrow&\frac{1}{8}(x_{k+1}-4)^2>0\\
        \Rightarrow&x_{k+1}<x_{k+2}\\
        &\text{Suppose: }x_{k+2}\geqslant4\\
        &\frac{1}{8}{x_{k+1}}^2+2\geqslant4\\
        &\frac{1}{8}{x_{k+1}}^2-2\geqslant0\\
        &\frac{1}{8}(x_{k+1}+4)(x_{k+1}-4)\geqslant0\\
        &x_{k+1}+4>0\\
        \Rightarrow&x_{k+1}-4\geqslant0\\
        &x_{k+1}\geqslant 4\nLeftrightarrow x_{k+1}<4\\
        \Rightarrow&x_{k+2}<4\\
        \Rightarrow&x_{k+1}<x_{k+2}<4\\
        &3<x_1<4\land (x_k<x_{k+1}<4\implies x_{k+1}<x_{k+2}<4)\\
        \Rightarrow&\forall n\in\mathbb{N} \cup\{0\}:x_n<x_{n+1}<4
    \end{align*}
\end{proof}

\newpage

\section*{Question 3}

~

\begin{proof}
    \begin{align*}
        &m=2\land k=1:\\
        &F_3=2\\
        &F_{1}F_1+F_2F_{2}=1+1=2\\
        \Rightarrow&F_3=F_1F_1+F_2F_2\\
        &\text{Suppose: }F_{p+q}=F_{p-1}F_q+F_pF_{q+1}\\
        &p+1:\\
        &F_{p+1+q}=F_{p+q}+F_{p-1+q}\\
        &F_{p+q}=F_{p-1}F_q+F_pF_{q+1}\\
        &F_{p-1+q}=F_{p-2}F_{q}+F_{p-1}F_{q+1}\\
        &F_{p+q}+F_{p-1+q}=F_{p-1}F_q+F_pF_{q+1}+F_{p-2}F_{q}+F_{p-1}F_{q+1}\\
        &=(F_{p-1}+F_{p-2})F_q+(F_p+F_{p-1})F_{q+1}\\
        &=F_pF_q+F_{p+1}F_{q+1}\\
        \Rightarrow&F_{p+1+q}=F_{p+1-1}F_q+F_{p+1}F_{q+1}\\
        &F_3=F_1F_1+F_2F_2\\
        &\land (F_{p+q}=F_{p-1}F_q+F_pF_{q+1}\implies F_{p+1+q}=F_{p+1-1}F_q+F_{p+1}F_{q+1})\\
        \Rightarrow&\forall m\in\mathbb{N},m\geqslant 2 :F_{m+q}=F_{m-1}F_q+F_mF_{q+1}\\
        &q+1:\\
        &F_{p+q+1}=F_{p+q}+F_{p+q-1}\\
        &F_{p+q}=F_{p-1}F_q+F_pF_{q+1}\\
        &F_{p+q-1}=F_{p-1}F_{q-1}+F_pF_q\\
        &F_{p+q}+F_{p+q-1}=F_{p-1}F_q+F_pF_{q+1}+F_{p-1}F_{q-1}+F_pF_q\\
        &=F_{p-1}(F_q+F_{q-1})+F_p(F_{q+1}+F_q)\\
        &=F_{p-1}F_{q+1}+F_pF_{q+2}\\
        \Rightarrow&F_{p+q+1}=F_{p-1}F_{q+1}+F_pF_{q+1+1}\\
        &F_3=F_1F_1+F_2F_2\\
        &\land (F_{p+q}=F_{p-1}F_q+F_pF_{q+1}\implies F_{p+q+1}=F_{p-1}F_{q+1}+F_pF_{q+1+1})\\
        \Rightarrow&\forall k\in\mathbb{N} ,F_{q+k}=F_{p-1}F_k+F_pF_{k+1}\\
        \Rightarrow&\forall m,k\in\mathbb{N} ,m\geqslant 2:F_{m+k}=F_{m-1}F_k+F_mF_{k+1}\\
    \end{align*}
\end{proof}

\newpage

\section*{Question 4}

~

\begin{proof}
    \begin{align*}
        &P_N\coloneqq \{a_kx^k+a_{k-1}x^{k-1}+...+a_1x+a_0=0|k+|a_k|+...+|a_0|=N\}\\
        &k,a_0,...,a_k\in\mathbb{Z}^+\\
        \Rightarrow&P_N\text{ is finite}\\
        &Z_N\coloneqq \{z|\exists p(x)\in P_N:p(z)=0\}\\
        &k\text{ is finite}\\
        \Rightarrow&\text{There are }k\text{ complex roots for polynomials of degree }k\\
        &P_N\text{ is finite}\\
        \Rightarrow&\text{Only finite combinations of }p(x)\in P_N\\
        \Rightarrow&\text{Finite roots for }p(x)\in P_N\\
        \Rightarrow&Z_N\text{ is finite}\\
        \Rightarrow&\bigcup_{N\geqslant 0}Z_N\text{ is countable}\\
        &\bigcup_{N\geqslant0}Z_N\text{ are all the algebraic numbers}\\
        \Rightarrow&\text{The set of algebraic numbers are countably infinite}\\
    \end{align*}
\end{proof}
\end{document}